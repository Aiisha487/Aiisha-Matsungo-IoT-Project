\documentclass[11pt]{article}

% Packages
\usepackage[margin=1in]{geometry}
\usepackage{hyperref}
\usepackage{booktabs}
\usepackage{longtable}
\usepackage{amsmath}
\usepackage{graphicx}
\usepackage{enumitem}

% Settings
\hypersetup{
    colorlinks=true,
    linkcolor=blue,
    urlcolor=blue,
    citecolor=blue
}

\title{\textbf{IoT-Based Acoustic Monitoring System\\for Designated Quiet Zones}\\
\vspace{0.5em}
\large Project Engineering Specification (PES) -- Version 1.0}

\author{
    \textbf{Student:} Aiisha Matsungo\\
    \textbf{NUID:} 002530298\\
    \textbf{Course:} EECE 5155 -- Wireless Sensor Networks and IoT Systems\\
    \textbf{Semester:} Spring 2026\\
    \textbf{Track:} Implementation\\
    \textbf{Date:} February 2026
}

\date{}

\begin{document}

\maketitle
\thispagestyle{empty}
\newpage

\tableofcontents
\newpage

% ============================================================================
% SECTION 1: PROBLEM DEFINITION
% ============================================================================
\section{Problem Definition and Use Case}
\label{sec:problem}

\subsection{Problem Statement}

Public facilities designate quiet zones for health activities (meditation, prayer, sensory regulation, focused work) requiring acoustic levels below 40-50 dB, but lack verification systems to ensure zones maintain standards. This results in:

\begin{enumerate}[noitemsep]
    \item \textbf{Poor placement decisions:} Zones near elevators or rail lines achieve only 50-65 dB baseline, failing to meet quiet standards
    \item \textbf{No user wayfinding:} People cannot locate or verify quiet zones before traveling to them, causing wasted time and failed health interventions
    \item \textbf{No compliance data:} Facilities cannot justify improvement investments (\$5,000-15,000 soundproofing) without objective measurements
\end{enumerate}

\subsection{Target Stakeholders}

\textbf{Primary users:} Students studying, people meditating/praying, persons with sensory sensitivities managing overwhelm, hospital visitors processing emotions, travelers needing rest.

\textbf{Secondary users:} Facility managers (libraries, hospitals, airports, hotels) needing compliance verification and placement optimization data.

\subsection{Current Baseline}

Facilities post ``Quiet Zone'' signs with no enforcement or verification. Staff patrol intermittently (reactive, cannot monitor all zones). Users must physically enter zones to discover if actually quiet, often finding spaces noisy (60-70 dB from conversations, poor placement). No data exists on zone effectiveness, placement quality, or utilization patterns.

\subsection{Why Wireless Sensor Network / IoT}

WSN appropriate because:
\begin{enumerate}[noitemsep]
    \item Multiple distributed zones require monitoring (wired impractical for retrofits)
    \item Existing facility WiFi infrastructure eliminates need for dedicated network
    \item Continuous 24/7 monitoring needed (manual staff checks infeasible)
    \item Real-time status display enables user wayfinding (immediate decision-making)
    \item Cloud aggregation allows facility-wide analytics (identify patterns across zones)
\end{enumerate}

\subsection{Alternatives Considered and Rejected}

\begin{enumerate}
    \item \textbf{Handheld sound level meters with manual logging}
    
    \textit{Rejected because:} Spot-checks miss temporal variations (zone quiet at 3 AM, loud at noon). Labor-intensive (staff time cost exceeds automated system). No real-time user information (cannot enable wayfinding).
    
    \item \textbf{Standalone battery-powered noise loggers}
    
    \textit{Rejected because:} No real-time display capability (data downloaded weekly). No network connectivity (cannot aggregate multi-zone status). Expensive (\$200-500 per unit vs \$30 for networked sensor). No user-facing interface.
\end{enumerate}

% ============================================================================
% SECTION 2: DEPLOYMENT SCENARIO
% ============================================================================
\section{Deployment Scenario and Scale}
\label{sec:deployment}

\subsection{Environment Description}

Target deployment: Designated quiet zones within institutional buildings (libraries, hospitals, airports). Indoor environment with HVAC control (15-30°C, 30-70\% RH). Acoustic challenges include external noise intrusion (rail lines, traffic: 50-70 dB), internal sources (elevators, HVAC: 45-60 dB), and user behavior (conversations: 60-75 dB). Building construction typically concrete/drywall with hard reflective surfaces. WiFi infrastructure present (802.11g/n). Standard 120V AC power outlets available.

\subsection{Network Topology}

\textbf{Topology:} Star topology with sensors connecting directly to existing facility WiFi access points, then to cloud via internet (no dedicated gateway for MVP).

\textbf{Justification:}
\begin{itemize}[noitemsep]
    \item Existing WiFi infrastructure eliminates gateway cost (\$150 savings)
    \item Star topology simplest for proof-of-concept
    \item Sensors statically placed (no mobility requires mesh)
    \item 2-3 sensors all within WiFi range (no multi-hop needed)
\end{itemize}

\textbf{Node placement:} One sensor per designated quiet zone, wall-mounted 4-5 ft height, central to room coverage area.

\subsection{Deployment Scale}

\begin{table}[htbp]
\centering
\caption{Deployment Scale and Assumptions}
\label{tab:deployment}
\begin{tabular}{@{}p{4.5cm}p{3cm}p{5cm}@{}}
\toprule
\textbf{Parameter} & \textbf{Value} & \textbf{Justification} \\
\midrule
Deployment Area & 800 m² & 2-3 zones × 250-300 m² each \\
Number of Sensor Nodes & 2-3 & One sensor per zone (MVP); scalable to 5-10 \\
Number of Gateways & 0 & Direct WiFi to cloud (existing infrastructure) \\
Environment Type & Indoor & Institutional buildings; HVAC controlled \\
Node Mobility & Static & Fixed mounting per zone \\
Duty Cycle & 100\% & Continuous monitoring (wall-powered) \\
Expected Lifetime & 2+ years & Electronic components MTBF >10,000 hrs \\
Operating Temperature & 15-30°C & Indoor HVAC; ESP8266 rated -40 to +125°C \\
\bottomrule
\end{tabular}
\end{table}

\subsection{Physical Site Survey}

A physical site survey was conducted at Northeastern University facilities and a local healthcare facility in January 2026 to characterize representative quiet zone environments.

\subsubsection{Locations Surveyed}

\textbf{Location 1: Snell Library Open Study Area}
\begin{itemize}[noitemsep]
    \item Construction: Glass boundaries, semi-open plan
    \item Floor: Commercial carpet (gray/blue pattern)
    \item Ceiling: Exposed painted ductwork with suspended LED fixtures
    \item Adjacent: High-traffic corridor with hard flooring
    \item Predicted baseline: 40-45 dB SPL
    \item Challenge level: Medium (glass boundaries allow corridor noise intrusion)
\end{itemize}

\textbf{Location 2: Snell Library 4th Floor Glass Room}
\begin{itemize}[noitemsep]
    \item Construction: Floor-to-ceiling glass enclosure (STC 26-28)
    \item Dimensions: Approximately 3m × 3m (9 m²)
    \item Occupancy: High density (5-6 students observed)
    \item Predicted baseline: 42-48 dB SPL
    \item Challenge level: \textbf{High} (glass + small volume + high occupancy)
    \item Risk identified: May exceed 45 dB threshold even during compliant behavior
\end{itemize}

\textbf{Location 3: Healthcare Facility ``Reflection Room''}
\begin{itemize}[noitemsep]
    \item Construction: Solid drywall (STC 33-37), French door
    \item Dimensions: Approximately 3m × 4m (12 m²)
    \item Adjacent: Low-traffic waiting area
    \item Predicted baseline: 35-40 dB SPL
    \item Challenge level: Low (good acoustic isolation)
\end{itemize}

\subsubsection{Key Findings}

\begin{table}[htbp]
\centering
\caption{Construction Type Comparison}
\label{tab:construction_comparison}
\begin{tabular}{@{}p{3cm}p{3cm}p{3cm}p{3.5cm}@{}}
\toprule
\textbf{Location} & \textbf{Wall Type} & \textbf{Baseline (est)} & \textbf{Challenge} \\
\midrule
Snell Open & Glass boundaries & 40-45 dB & Medium \\
Snell Glass Room & Full glass (STC 26-28) & 42-48 dB & High \\
Healthcare & Solid (STC 33-37) & 35-40 dB & Low \\
\bottomrule
\end{tabular}
\end{table}

\textbf{Noise sources identified:}
\begin{itemize}[noitemsep]
    \item HVAC (ceiling vents): 38-42 dB continuous
    \item Corridor footsteps: 45-55 dB intermittent (5-10 events/min)
    \item Door operation: 50-60 dB peak (<30 second duration)
    \item Corridor conversations (transmitted through glass): 50-60 dB variable
    \item In-room conversations (violation target): 60-70 dB
\end{itemize}

\textbf{Infrastructure verification:}
\begin{itemize}[noitemsep]
    \item WiFi: Northeastern enterprise WiFi (eduroam/NUwave) confirmed available
    \item Signal strength: Estimated RSSI -45 to -55 dBm (excellent)
    \item Power: 120V AC outlets accessible in corridors and rooms
    \item Mounting: Drywall sections suitable for damage-free mounting (3M Command strips)
\end{itemize}

% ============================================================================
% SECTION 3: SIGNALS, SENSORS, AND PROCESSING
% ============================================================================
\section{Signals, Sensors, and Processing Platform}
\label{sec:signals}

\subsection{Measured Physical Quantity}

\textbf{Physical phenomenon:} Sound pressure (acoustic pressure waves)

\textbf{SI unit:} Pascal (Pa) = Newton/meter² (N/m²)

\textbf{Practical unit:} Decibels Sound Pressure Level (dB SPL)

\textbf{Definition:}
\begin{equation}
\text{SPL (dB)} = 20 \cdot \log_{10}\left(\frac{P_{\text{rms}}}{P_{\text{ref}}}\right)
\end{equation}

where $P_{\text{rms}}$ is root-mean-square sound pressure and $P_{\text{ref}} = 20$ µPa (threshold of human hearing at 1 kHz).

\textbf{Measurement range:} 35-100 dB SPL (quiet room to loud conversation)

\textbf{Target accuracy:} ±3 dB (after calibration)

\subsection{Signal Chain}

\begin{verbatim}
Sound Pressure (Pa)
  ↓ [MAX9814 Electret Microphone]
Analog Voltage (0-3.3V)
  ↓ [ESP8266 10-bit ADC, 1 kHz sampling]
Digital Samples (0-1023)
  ↓ [RMS Calculation: 1-second window, 1000 samples]
Sound Pressure Level (dB SPL, 1 Hz output)
  ↓ [10-sample Moving Average Filter]
Smoothed dB Level (0.1 Hz)
  ↓ [WiFi HTTP POST, 30-second reporting]
ThingSpeak Cloud Time-Series Database
\end{verbatim}

\subsection{Sensor Selection}

\subsubsection{Microphone: MAX9814 Electret with AGC}

\textbf{Selected component:} Adafruit MAX9814 Microphone Breakout (Product \#1713)

\textbf{Specifications:}
\begin{itemize}[noitemsep]
    \item Transduction: Electret condenser (acoustic pressure → capacitance → voltage)
    \item Output: Analog voltage (0-3.3V)
    \item Signal-to-Noise Ratio: 58 dB
    \item Sensitivity: -44 dBV/Pa
    \item Frequency response: 100 Hz - 10 kHz (±3 dB)
    \item Power: 3 mA @ 3.3V = 10 mW
    \item Gain control: Integrated automatic gain control (AGC)
    \item Cost: \$10
\end{itemize}

\subsubsection{Alternative Considered}

\textbf{SPH0645LM4H-B I2S MEMS Microphone (Adafruit \#3421)}
\begin{itemize}[noitemsep]
    \item SNR: 65 dB (7 dB better than MAX9814)
    \item Output: I2S digital (24-bit samples)
    \item Cost: \$7 (\$3 cheaper)
\end{itemize}

\subsubsection{Decision Matrix}

\begin{table}[htbp]
\centering
\caption{Microphone Selection Trade-off}
\label{tab:mic_selection}
\begin{tabular}{@{}p{3.5cm}p{2.5cm}p{2.5cm}p{3.5cm}@{}}
\toprule
\textbf{Factor} & \textbf{MAX9814} & \textbf{SPH0645} & \textbf{Winner} \\
\midrule
Implementation time & 2 hours & 8 hours & MAX9814 (-6 hr) \\
Debug risk & Low & Medium-High & MAX9814 \\
SNR & 58 dB & 65 dB & SPH0645 (+7 dB) \\
Cost & \$10 & \$7 & SPH0645 (-\$3) \\
Meets ±3 dB target? & Yes & Yes & Tie \\
\midrule
\textbf{Decision} & \multicolumn{3}{c}{\textbf{MAX9814 selected}} \\
\bottomrule
\end{tabular}
\end{table}

\textbf{Selection justification:}

\begin{enumerate}[noitemsep]
    \item \textbf{SNR difference irrelevant for application:} Both microphones exceed the 45 dB detection threshold with adequate margin. MAX9814's 58 dB SNR → 36 dB noise floor → 9 dB margin above 45 dB target. SPH0645's 7 dB SNR advantage provides no functional benefit for 35-100 dB detection range.
    
    \item \textbf{Implementation risk:} I2S protocol requires clock configuration, DMA setup, and buffer management (50+ lines of code, 4-8 hour debug time). Analog path uses single \texttt{analogRead()} call (<1 hour implementation). For 10-week project timeline, 6-hour time savings represents significant schedule risk reduction.
    
    \item \textbf{Trade-off accepted:} 7 dB worse SNR + \$3 higher cost in exchange for 6-hour time savings and lower implementation risk. This represents mature engineering judgment: optimize the constraint that matters (schedule, risk) not the spec that doesn't (SNR beyond requirement).
\end{enumerate}

\subsubsection{Feasibility Check: Noise Floor vs Detection Threshold}

From MAX9814 datasheet:

SNR = 58 dB at 94 dB SPL reference

Therefore: Noise Floor = 94 dB - 58 dB = 36 dB SPL equivalent

Detection margin at 45 dB threshold:

Margin = 45 dB (signal) - 36 dB (noise) = \textbf{9 dB} ✓

\textbf{Verdict:} Feasible. 9 dB margin provides adequate signal-to-noise for reliable detection. Rule of thumb: Need >6 dB SNR for <1\% error rate in amplitude measurements.

\subsection{Processing Platform Selection}

\subsubsection{Microcontroller: ESP8266}

\textbf{Selected platform:} ESP8266 NodeMCU v3 Development Board

\textbf{Specifications:}
\begin{itemize}[noitemsep]
    \item MCU: ESP8266EX @ 80 MHz (Tensilica L106 32-bit RISC)
    \item RAM: 80 KB user-available (160 KB total, 80 KB reserved for WiFi stack)
    \item Flash: 4 MB external SPI
    \item ADC: 10-bit SAR (0-1023), 0-1V input (3.3V via voltage divider)
    \item WiFi: 802.11 b/g/n, 2.4 GHz, -97 dBm sensitivity
    \item GPIO: 11 pins available after WiFi/UART
    \item Power: 80 mA @ 3.3V typical (WiFi active)
    \item Cost: \$6
\end{itemize}

\textbf{Alternative considered:} ESP32 (dual-core, I2S support, \$6)

\textbf{Decision:} ESP8266 selected. Single-core adequate for RMS calculation (<0.1\% CPU load). No I2S needed (using analog microphone). 80 KB RAM sufficient for WiFi/TCP stack. Same cost as ESP32; simpler platform reduces learning curve.

\subsection{Sampling Rate Determination}

\subsubsection{Signal Bandwidth Analysis}

Target acoustic signals:
\begin{itemize}[noitemsep]
    \item Human speech fundamental: 85-250 Hz
    \item Speech harmonics: 250 Hz - 4 kHz
    \item HVAC noise: 50-500 Hz (broadband)
    \item Footsteps/doors: 100 Hz - 2 kHz (transients)
\end{itemize}

\textbf{Target bandwidth:} DC to 4 kHz (captures speech and ambient noise)

\subsubsection{Nyquist Criterion}

Per Shannon-Nyquist sampling theorem:
\begin{equation}
f_s \geq 2 \cdot f_{\text{max}}
\end{equation}

where $f_{\text{max}}$ is the highest frequency of interest.

\textbf{Minimum sampling rate:}
\begin{equation}
f_{s,\text{min}} = 2 \times 4000\text{ Hz} = 8000\text{ Hz}
\end{equation}

\textbf{Selected sampling rate: 1000 Hz (1 kHz)}

\subsubsection{Justification for Sampling Below Nyquist Rate}

\textbf{Why 1 kHz is acceptable despite violating Nyquist:}

\begin{enumerate}[noitemsep]
    \item \textbf{Application does not require frequency discrimination}
    
    Goal: Measure \textit{amplitude} (dB SPL), not frequency content. No need to distinguish speech from noise (both violations if loud). No speech recognition, no FFT, no spectral analysis required.
    
    \item \textbf{Anti-aliasing not critical for amplitude measurement}
    
    Aliasing distorts frequency representation (high-frequency signals ``fold'' into low-frequency band). However, aliasing does NOT distort amplitude/energy measurement. Parseval's theorem: energy conserved under aliasing.
    
    For RMS calculation:
    \begin{equation}
    \text{RMS} = \sqrt{\frac{1}{N}\sum_{i=1}^{N} v_i^2}
    \end{equation}
    
    Result identical whether signal is at true frequency or aliased frequency, because total energy unchanged.
    
    \item \textbf{Computational efficiency}
    
    1 kHz sampling: 1000 samples/second × 10-bit = 10 kb/s data rate
    
    8 kHz sampling: 8000 samples/second × 10-bit = 80 kb/s data rate (8× processing load)
    
    ESP8266 has ample capacity for either, but no benefit from higher rate.
    
    \item \textbf{When 1 kHz would NOT work}
    
    Only if future enhancements require frequency-domain features:
    \begin{itemize}[noitemsep]
        \item Speech vs non-speech classification (ML distinguishing conversations from HVAC)
        \item Acoustic signature identification (footsteps vs door slams via spectral analysis)
        \item Occupancy estimation via voice counting (requires preserved harmonics)
    \end{itemize}
\end{enumerate}

\textbf{Conclusion:} For amplitude-only monitoring, 1 kHz sampling sufficient.

\subsection{Multi-Rate Sampling Architecture}

\begin{table}[htbp]
\centering
\caption{Sampling Rate Hierarchy}
\label{tab:sampling_hierarchy}
\begin{tabular}{@{}p{3.5cm}p{2cm}p{6cm}@{}}
\toprule
\textbf{Stage} & \textbf{Rate} & \textbf{Justification} \\
\midrule
Raw ADC sampling & 1 kHz & Amplitude measurement (no frequency analysis needed) \\
RMS calculation & 1 Hz & 1-second window captures door slams, smooths footsteps \\
Moving average filter & 0.1 Hz & 10-second smoothing rejects transients \\
Cloud transmission & 1/30 Hz & 30-second interval balances latency vs API limits \\
\midrule
\textbf{Data reduction} & \textbf{30,000×} & 1000 Hz raw → 0.033 Hz transmitted \\
\bottomrule
\end{tabular}
\end{table}

\subsection{Measurement Accuracy Analysis}

\subsubsection{Error Budget}

Target accuracy: ±3 dB across 35-100 dB SPL range

\textbf{Error sources (RSS combination):}

\begin{table}[htbp]
\centering
\caption{Measurement Error Budget}
\label{tab:error_budget}
\begin{tabular}{@{}p{5cm}p{2.5cm}p{4cm}@{}}
\toprule
\textbf{Error Source} & \textbf{Magnitude} & \textbf{Justification} \\
\midrule
Microphone tolerance & ±1 dB & MAX9814 datasheet typical \\
ADC quantization (10-bit) & ±0.5 dB & 0.06 dB/step negligible \\
Calibration curve fit & ±1 dB & 6-point linear regression \\
Temperature drift (15-30°C) & ±0.5 dB & Indoor range minimal \\
\midrule
\textbf{RSS Total} & \textbf{±1.6 dB} & $\sqrt{1^2 + 0.5^2 + 1^2 + 0.5^2}$ \\
\textbf{Design Target} & \textbf{±3 dB} & 1.9× margin \\
\bottomrule
\end{tabular}
\end{table}

\textbf{Verdict:} Predicted 1.6 dB < 3 dB target. Conservative design accounts for unknowns.

\subsubsection{Calibration Procedure}

\textbf{Reference equipment:} BAFX3608 Digital Sound Level Meter (ANSI S1.4 Class 2, ±1.5 dB, \$30)

\textbf{Calibration steps:}
\begin{enumerate}[noitemsep]
    \item Position reference meter and ESP8266 sensor 1m from speaker
    \item Generate 1 kHz tone, adjust volume to 40 dB (per reference meter)
    \item Record ESP8266 raw ADC output and reference meter reading
    \item Repeat for 50, 60, 70, 80, 94 dB levels (6 points total)
    \item Perform linear regression: $\text{dB}_{\text{cal}} = m \cdot \text{ADC} + b$
    \item Store calibration coefficients (m, b) in ESP8266 EEPROM
    \item Validate with intermediate levels (45, 65, 85 dB)
\end{enumerate}

\textbf{Recalibration schedule:} Semi-annual (every 6 months)

\subsection{Signal Specifications Summary}

\begin{table}[htbp]
\centering
\caption{Complete Signal Specifications}
\label{tab:signal_specs}
\begin{tabular}{@{}ll@{}}
\toprule
\textbf{Parameter} & \textbf{Value} \\
\midrule
Physical Quantity & Sound Pressure (Pa) \\
Measurement Unit & Decibels SPL (dB) \\
Sensor & MAX9814 Electret Microphone \\
Sensor Output & Analog voltage (0-3.3V) \\
ADC Resolution & 10-bit (1024 levels) \\
Measurement Resolution & 0.06 dB per ADC step \\
Target Accuracy & ±3 dB (calibrated) \\
Measurement Range & 35-100 dB SPL \\
Raw Sampling Rate & 1 kHz (analogRead()) \\
Processing Window & 1 second (RMS averaging) \\
Output Rate & 1 Hz (dB values) \\
Smoothing Filter & 10-sample moving average \\
Transmission Rate & 30 seconds (WiFi to cloud) \\
Data Format & JSON: zone\_id, timestamp, db\_spl, status \\
\bottomrule
\end{tabular}
\end{table}

% ============================================================================
% SECTION 4: SYSTEM CONSTRAINTS
% ============================================================================
\section{System Constraints Analysis}
\label{sec:constraints}

\subsection{Prioritized Constraints}

Constraints ranked by impact on design decisions:

\begin{table}[htbp]
\centering
\caption{Prioritized System Constraints}
\label{tab:constraints}
\small
\begin{tabular}{@{}cp{3cm}p{3cm}p{5cm}@{}}
\toprule
\textbf{Rank} & \textbf{Constraint} & \textbf{Target} & \textbf{Design Impact} \\
\midrule
\textbf{1} & Cost & <\$200 total & ESP8266 (\$6) not RPi (\$35); WiFi reuse (no \$150 gateway); 2-3 sensors only \\
\textbf{2} & Accuracy & ±3 dB SPL & Requires calibration; 58 dB SNR minimum; reference meter needed \\
\textbf{3} & Timeline & 10 weeks & Analog mic (2hr) not I2S (8hr risk); existing WiFi not custom network \\
\textbf{4} & Latency & <120 seconds & 30-sec reporting; always-on WiFi; adequate for wayfinding \\
\textbf{5} & Power & Wall-powered & 440 mW from USB; unconstrained (100\% duty cycle) \\
\textbf{6} & Network & WiFi available & 0.008\% utilization; ThingSpeak free tier \\
\textbf{7} & Environment & Indoor 15-30°C & Minimal protection; ±0.5 dB temp drift \\
\textbf{8} & Coverage & 4×5m room & 1 sensor adequate; 5m = 1 dB margin \\
\textbf{9} & Reliability & >90\% uptime & WiFi auto-reconnect; watchdog timer \\
\bottomrule
\end{tabular}
\end{table}

\subsection{Constraint Analysis}

\subsubsection{Constraint \#1: Cost (Highest Priority)}

\textbf{Target:} Total system <\$200 for 2-3 sensor MVP

\textbf{Why highest priority:} Student project budget constraint (fixed, non-negotiable). Proof-of-concept scope. Cost drives ALL major component decisions.

\textbf{Cost-driven decisions:}

\begin{itemize}[noitemsep]
    \item ESP8266 (\$6) vs Raspberry Pi (\$35): Saves \$87 for 3 nodes
    \item WiFi reuse vs LoRa gateway: Saves \$150
    \item Consumer meter (\$30) vs precision (\$300): Saves \$270
    \item Plastic enclosures (\$8) vs metal (\$25): Saves \$51 for 3 nodes
\end{itemize}

Total savings from cost-conscious decisions: \$558

\subsubsection{Constraint \#2: Measurement Accuracy}

\textbf{Target:} ±3 dB SPL across 35-100 dB range

\textbf{Requirement derivation:}

Baseline: 45 dB (compliant) → Measurement: 42-48 dB (±3 dB)

Violation: 60 dB (conversation) → Measurement: 57-63 dB (±3 dB)

Separation: 60 - 45 = 15 dB >> 2×3 dB = 6 dB total uncertainty

False alarm probability: $P(\text{overlap}) < 1\%$ with ±3 dB Gaussian error

\textbf{If accuracy were ±5 dB:} 45±5 = 40-50 dB; 60±5 = 55-65 dB; only 10 dB separation = 2σ → 2.3\% overlap → unacceptable false alarm rate

\textbf{Verdict:} ±3 dB is minimum acceptable (distinguishes states) and maximum achievable (within \$200 budget)

\subsubsection{Constraint \#3: Implementation Feasibility}

\textbf{Target:} Complete within 10-week semester timeline

\textbf{Timeline drives technology choices:}

\begin{table}[htbp]
\centering
\caption{Implementation Risk vs Timeline}
\label{tab:implementation_risk}
\begin{tabular}{@{}p{3.5cm}p{2cm}p{2cm}p{3cm}@{}}
\toprule
\textbf{Design Choice} & \textbf{Time} & \textbf{Risk} & \textbf{Decision} \\
\midrule
Analog mic (MAX9814) & 2 hours & Low & ✓ Selected \\
I2S mic (SPH0645) & 8+ hours & High & ✗ Rejected \\
WiFi (ESP8266) & 4 hours & Low & ✓ Selected \\
LoRa + gateway & 16+ hours & High & ✗ Rejected \\
ThingSpeak API & 2 hours & Low & ✓ Selected \\
Custom backend & 40+ hours & High & ✗ Rejected \\
\bottomrule
\end{tabular}
\end{table}

Critical path: Weeks 1-4 (component delivery + firmware). Buffer: 2 weeks.

\subsection{Constraint Trade-offs}

\begin{table}[htbp]
\centering
\caption{Constraint Conflicts and Resolutions}
\label{tab:tradeoffs}
\small
\begin{tabular}{@{}p{3.5cm}p{3.5cm}p{4.5cm}@{}}
\toprule
\textbf{Conflict} & \textbf{Trade-off} & \textbf{Resolution} \\
\midrule
Cost vs Accuracy & Cheap mic (\$5, 48 dB SNR) vs quality (\$10, 58 dB SNR) & Select \$10 mic; 58 dB required for ±3 dB target \\
Timeline vs Accuracy & I2S (±2 dB, 8hr) vs analog (±5 dB, 2hr) & Select analog; both meet ±3 dB; schedule risk > accuracy gain \\
Latency vs API Limits & 10-sec (60s latency) vs 30-sec (90s latency) & Select 30-sec; 10-sec exceeds free tier \\
\bottomrule
\end{tabular}
\end{table}

% ============================================================================
% SECTION 5: TRAFFIC AND CAPACITY
% ============================================================================
\section{Traffic and Capacity Estimation}
\label{sec:traffic}

\subsection{Packet Structure}

\subsubsection{Application Layer: JSON Payload}

\begin{verbatim}
{
  "zone_id": "snell_floor2_silent",
  "timestamp": 1706745600,
  "db_spl": 42.3,
  "status": "compliant"
}
\end{verbatim}

Payload size: 94 bytes (uncompressed, no whitespace)

\subsubsection{HTTP Request Headers}

\begin{verbatim}
POST /update?api_key=XXXXXXXXXXXXXXXX HTTP/1.1
Host: api.thingspeak.com
Content-Type: application/json
Content-Length: 94
Connection: close
\end{verbatim}

HTTP headers: 169 bytes

Total application layer: 263 bytes (169 + 94)

HTTP overhead ratio: 169 / 94 = 1.8× (headers add 180\% overhead)

\subsubsection{Complete Protocol Stack}

\begin{table}[htbp]
\centering
\caption{Protocol Overhead Breakdown}
\label{tab:protocol_overhead}
\begin{tabular}{@{}p{4.5cm}p{2.5cm}p{5cm}@{}}
\toprule
\textbf{Layer} & \textbf{Size (bytes)} & \textbf{Content} \\
\midrule
\multicolumn{3}{l}{\textit{TCP Connection Establishment (3 packets):}} \\
\quad WiFi + IP + TCP headers & 270 & SYN, SYN-ACK, ACK \\
\multicolumn{3}{l}{\textit{HTTP POST Request (1 packet):}} \\
\quad WiFi header & 38 & 802.11 frame \\
\quad IP header & 20 & IPv4 \\
\quad TCP header & 32 & Sequence, ACK \\
\quad HTTP headers & 169 & POST, Host, Content-Type \\
\quad JSON payload & 94 & \textbf{Actual data} \\
\quad Subtotal & 353 & Data transmission \\
\multicolumn{3}{l}{\textit{HTTP Response (1 packet from server):}} \\
\quad WiFi + IP + TCP + HTTP & 260 & Server acknowledgment \\
\multicolumn{3}{l}{\textit{TCP Connection Teardown (2 packets):}} \\
\quad WiFi + IP + TCP & 180 & FIN-ACK frames \\
\midrule
\textbf{Total bytes transmitted} & \textbf{1,063} & All 7 packets \\
\textbf{Actual payload} & \textbf{94} & What we wanted to send \\
\textbf{Protocol overhead} & \textbf{969} & Everything else \\
\textbf{Overhead ratio} & \textbf{10.3×} & 969 / 94 \\
\bottomrule
\end{tabular}
\end{table}

\subsection{Traffic Calculations}

\textbf{Per-node data rate:}

\begin{equation}
\text{Rate} = \frac{1063\text{ bytes}}{30\text{ s}} = 35.4\text{ bytes/s} = 283\text{ bps}
\end{equation}

\textbf{Aggregate (3 nodes):}

\begin{equation}
\text{Total} = 3 \times 283\text{ bps} = 849\text{ bps}
\end{equation}

\textbf{WiFi capacity (802.11g minimum):} 11 Mbps

\textbf{Utilization:}

\begin{equation}
\frac{849\text{ bps}}{11{,}000{,}000\text{ bps}} = 0.0077\%
\end{equation}

Still negligible (13,000× below capacity).

\textbf{Daily data volume:}

\begin{equation}
35.4\text{ bytes/s} \times 86{,}400\text{ s/day} \times 3\text{ nodes} = 9.2\text{ MB/day}
\end{equation}

Monthly: 276 MB/month

\textbf{Cloud API limit check (ThingSpeak):}

\begin{equation}
\frac{86{,}400\text{ s/day}}{30\text{ s}} \times 365\text{ days} \times 3\text{ nodes} = 3{,}153{,}600\text{ msg/year}
\end{equation}

Free tier limit: 3,000,000 msg/year per channel

Utilization: 105\% → Slightly over limit

\textbf{Mitigation:} Use 3 separate ThingSpeak channels (1 per sensor) = 9M capacity total ✓

\subsection{Traffic Budget Summary}

\begin{table}[htbp]
\centering
\caption{Traffic Budget}
\label{tab:traffic}
\begin{tabular}{@{}p{4.5cm}p{2.5cm}p{5cm}@{}}
\toprule
\textbf{Parameter} & \textbf{Value} & \textbf{Calculation} \\
\midrule
Payload size (JSON) & 94 bytes & Actual data \\
Total packet & 1,063 bytes & With protocol overhead \\
Reporting interval & 30 seconds & Balance latency vs API \\
Per-node rate & 283 bps & 1063 bytes / 30s × 8 \\
Node count & 3 & 2-3 zones \\
Aggregate rate & 849 bps & 3 × 283 \\
WiFi utilization & 0.0077\% & 849 / 11M \\
Daily volume & 9.2 MB & 35.4 bytes/s × 86400 × 3 \\
\midrule
\textbf{Verdict} & \multicolumn{2}{l}{Bandwidth NOT a constraint} \\
\bottomrule
\end{tabular}
\end{table}

% ============================================================================
% SECTION 6: OPERATIONAL MODEL AND BOM
% ============================================================================
\section{Operational Model and Bill of Materials}
\label{sec:bom}

\subsection{System Operation Flow}

\begin{verbatim}
Sense → MAX9814 (acoustic pressure → voltage)
  ↓
Sample → ESP8266 ADC (1 kHz, 10-bit)
  ↓
Process → RMS (1-sec window) → dB conversion (calibration) 
         → moving average (10-sample)
  ↓
Transmit → WiFi HTTP POST (every 30 seconds)
  ↓
Store → ThingSpeak time-series database
  ↓
Display → Web dashboard (real-time status, green/red)
  ↓
Alert → Email if sustained violation (>5 minutes)
\end{verbatim}

\subsection{Power Budget}

\begin{table}[htbp]
\centering
\caption{System Power Consumption}
\label{tab:power_budget}
\begin{tabular}{@{}p{5cm}p{2.5cm}p{3cm}@{}}
\toprule
\textbf{Component} & \textbf{Current} & \textbf{Power} \\
\midrule
ESP8266 (WiFi active) & 80 mA @ 3.3V & 264 mW \\
MAX9814 microphone & 3 mA @ 3.3V & 10 mW \\
AMS1117 LDO regulator & 5 mA @ 5V & 25 mW \\
LDO dropout loss & (5-3.3V) × 83 mA & 141 mW \\
\midrule
\textbf{Total} & \textbf{88 mA @ 5V} & \textbf{440 mW} \\
\midrule
USB capacity & 500 mA @ 5V & 2500 mW \\
Utilization & 18\% & 5.7× margin \\
\midrule
Annual energy & & 3.85 kWh/year \\
Annual cost & & \$0.58 @ \$0.15/kWh \\
\bottomrule
\end{tabular}
\end{table}

\subsection{Bill of Materials}

\begin{table}[htbp]
\centering
\caption{Detailed Bill of Materials}
\label{tab:bom}
\begin{tabular}{@{}p{5cm}p{2.5cm}rr@{}}
\toprule
\textbf{Component} & \textbf{Supplier/Part \#} & \textbf{Unit} & \textbf{Total} \\
\midrule
\multicolumn{4}{l}{\textit{\textbf{Sensor Nodes (3×):}}} \\
\quad ESP8266 NodeMCU v3 & Amazon/AliExpress & \$6 × 3 & \$18 \\
\quad MAX9814 Mic Breakout & Adafruit \#1713 & \$10 × 3 & \$30 \\
\quad USB Cable (6ft) & AmazonBasics & \$3 × 3 & \$9 \\
\quad 5V 1A Wall Adapter & Generic/Anker & \$2 × 3 & \$6 \\
\quad Mounting (Command Strips) & 3M & \$2 × 3 & \$6 \\
\midrule
\multicolumn{3}{l}{\textbf{Sensor Nodes Subtotal:}} & \textbf{\$69} \\
\midrule
\multicolumn{4}{l}{\textit{\textbf{Validation \& Development:}}} \\
\quad Sound Level Meter & BAFX3608 (Amazon) & \$30 × 1 & \$30 \\
\quad Breadboard + Jumpers & Electronics Kit & \$10 × 1 & \$10 \\
\quad Multimeter (optional) & Harbor Freight & \$15 × 1 & \$15 \\
\midrule
\multicolumn{3}{l}{\textbf{Validation Subtotal:}} & \textbf{\$55} \\
\midrule
\multicolumn{4}{l}{\textit{\textbf{Optional (Contingency):}}} \\
\quad Enclosures (ABS plastic) & Hammond 1591 & \$8 × 3 & \$24 \\
\quad Spare ESP8266 & Backup unit & \$6 × 1 & \$6 \\
\quad Spare MAX9814 & Backup unit & \$10 × 1 & \$10 \\
\midrule
\multicolumn{3}{l}{\textbf{Optional Subtotal:}} & \textbf{\$40} \\
\midrule
\midrule
\multicolumn{3}{l}{\textbf{TOTAL (Core System):}} & \textbf{\$124} \\
\multicolumn{3}{l}{\textbf{TOTAL (With Optional):}} & \textbf{\$164} \\
\multicolumn{3}{l}{\textbf{Budget Remaining:}} & \textbf{\$36-76} \\
\bottomrule
\end{tabular}
\end{table}

\subsection{Maintenance Schedule}

\begin{table}[htbp]
\centering
\caption{Maintenance Requirements}
\label{tab:maintenance}
\begin{tabular}{@{}p{3cm}p{5cm}p{2.5cm}@{}}
\toprule
\textbf{Frequency} & \textbf{Task} & \textbf{Duration} \\
\midrule
Weekly & Dashboard review, verify sensors online & 5 min \\
Monthly & Spot-check calibration (1 sensor vs meter) & 30 min \\
Semi-annual & Full recalibration (6-point, all sensors) & 3 hours \\
Annual & Hardware inspection, component replacement & 1-2 hours \\
As needed & Power cycle / hardware swap & 15 min \\
\midrule
\textbf{Total} & & \textbf{\textasciitilde 15 hours/year} \\
\bottomrule
\end{tabular}
\end{table}

% ============================================================================
% SECTION 7: RISKS AND ASSUMPTIONS
% ============================================================================
\section{Risks, Assumptions, and Unknowns}
\label{sec:risks}

\subsection{Technical Risks}

\begin{table}[htbp]
\centering
\caption{Risk Assessment Matrix}
\label{tab:risks}
\small
\begin{tabular}{@{}p{3.5cm}p{1.5cm}p{1.5cm}p{5cm}@{}}
\toprule
\textbf{Risk} & \textbf{Likelihood} & \textbf{Impact} & \textbf{Mitigation} \\
\midrule
Glass room baseline >45 dB & High & Medium & Zone-specific threshold (48-50 dB); flag as non-compliant; use data to recommend improvements \\
Calibration fails ±3 dB target & Medium & High & NIST-traceable meter; controlled environment; 6-point fit; fallback: relative measurements \\
WiFi connectivity unreliable & Low & High & Pre-test coverage; retry logic (3×); fallback: phone hotspot \\
False alarm rate >10\% & Medium & Medium & 5-min sustained violation; 10-sec filter; iterative tuning \\
\bottomrule
\end{tabular}
\end{table}

\subsection{Failure Mode and Effects Analysis}

\begin{table}[htbp]
\centering
\caption{FMEA Summary}
\label{tab:fmea}
\scriptsize
\begin{tabular}{@{}p{2.5cm}p{1.3cm}p{1.3cm}p{1.8cm}p{3cm}@{}}
\toprule
\textbf{Failure Mode} & \textbf{Prob} & \textbf{Impact} & \textbf{Detection} & \textbf{Mitigation} \\
\midrule
WiFi outage & Med & High & HTTP timeout & Auto-reconnect; 3× retry; local buffer (1hr); watchdog \\
Mic failure & Low & High & Flatline & Heartbeat self-test; alert if >5 min; spare swap (\$10) \\
Power loss & Low & High & No heartbeat & Accept gap (monitoring not critical control) \\
Cal drift & Med & Med & Validation & Semi-annual recal; store history; flag >5 dB deviation \\
False alarm & High & Low & User feedback & 5-min sustained + 10-sec filter; zone-specific thresholds \\
\bottomrule
\end{tabular}
\end{table}

\subsection{Key Assumptions}

\begin{enumerate}[noitemsep]
    \item \textbf{WiFi infrastructure available in target facilities}
    
    Consequence if false: Must add gateway (\$150+ cost). Confidence: High (validated by site survey). Mitigation: Phone hotspot fallback for testing.
    
    \item \textbf{Quiet zone acoustic standard is 40-50 dB}
    
    Consequence if false: If <35 dB required, sensor noise floor (36 dB) becomes limiting. Confidence: Medium-High (ANSI S12.60 specifies 35 dB classrooms; libraries 40-50 dB typical). Mitigation: Document limitation; recommend precision mic for <40 dB applications.
    
    \item \textbf{Power outlets accessible in quiet zones}
    
    Consequence if false: Use battery (8-hour operation, weekly recharge burden). Confidence: High (modern construction standard). Mitigation: Long USB cables (10 ft) extend reach.
    
    \item \textbf{Lab/bench testing representative of facility deployment}
    
    Consequence if false: Field deployment reveals unexpected issues. Confidence: Medium (validates technical function but not behavioral aspects). Mitigation: Phased approach (lab → limited facility → full deployment).
\end{enumerate}

% ============================================================================
% SECTION 8: MVP SCOPE
% ============================================================================
\section{MVP Scope and Success Criteria}
\label{sec:mvp}

\subsection{In Scope (Must Have)}

\begin{itemize}[noitemsep]
    \item 2 sensor nodes (Snell Library open area + Healthcare Reflection Room)
    \item MAX9814 analog microphones + ESP8266
    \item WiFi connectivity to ThingSpeak (2 channels, free tier)
    \item Basic web dashboard (real-time status display, green/red per zone)
    \item Calibration with reference meter (6-point procedure)
    \item 48-hour continuous operation validation (minimum)
    \item Controlled scenario testing:
    \begin{itemize}[noitemsep]
        \item Baseline characterization (empty room overnight)
        \item Compliance detection (40 dB sustained)
        \item Violation detection (60 dB conversation)
        \item Transient filtering (door slam, <10 sec duration)
        \item Placement quality comparison (2 zones, different baselines)
    \end{itemize}
    \item Technical report documenting design trade-offs and validation results
\end{itemize}

\subsection{Out of Scope}

\begin{itemize}[noitemsep]
    \item Frequency analysis (FFT, spectrograms, speech recognition)
    \item Machine learning (classification, prediction, pattern detection)
    \item Mobile apps (iOS/Android native applications)
    \item Actual facility deployment (if permission not obtained by Week 6)
    \item SMS/push notifications (email alerts acceptable for MVP)
    \item Gateway architecture (direct WiFi sufficient for 2-3 sensors)
    \item Multi-facility scaling (single proof-of-concept validates concept)
\end{itemize}

\subsection{Stretch Goals (If Time Permits by Week 6)}

\begin{itemize}[noitemsep]
    \item 3rd sensor node (Snell 4th floor glass room - worst-case validation)
    \item Zone-specific threshold tuning (45 dB solid walls, 48-50 dB glass rooms)
    \item Email alerts on sustained violations (>5 minutes above threshold)
    \item 2-week continuous operation data collection (vs 48-hour minimum)
    \item Actual Snell Library deployment (if permission granted)
\end{itemize}

\subsection{Success Criteria}

\begin{table}[htbp]
\centering
\caption{MVP Success Criteria}
\label{tab:success_criteria}
\begin{tabular}{@{}p{4cm}p{3cm}p{4.5cm}@{}}
\toprule
\textbf{Criterion} & \textbf{Target} & \textbf{Measurement Method} \\
\midrule
Measurement accuracy & ±3 dB SPL & 6-point calibration validation \\
End-to-end latency & <120 seconds & Event timestamp → dashboard \\
State discrimination & Detect 45 vs 60 dB & 0\% false pos/neg (10 min each) \\
System uptime & >90\% (48 hours) & Heartbeat gap analysis \\
Coverage & 5m max distance & Propagation test: 1 dB margin \\
Transient filtering & No false alarms & 5× door slam: 0 alerts \\
Placement detection & Distinguish zones & 10-15 dB baseline difference \\
\bottomrule
\end{tabular}
\end{table}

\subsection{Validation Test Plan}

\begin{table}[htbp]
\centering
\caption{Validation Tests with Acceptance Criteria}
\label{tab:validation}
\scriptsize
\begin{tabular}{@{}p{1.3cm}p{3cm}p{3cm}p{3.5cm}@{}}
\toprule
\textbf{Test ID} & \textbf{Procedure} & \textbf{Expected Result} & \textbf{Pass Criteria} \\
\midrule
VAL-1 & Calibration accuracy: 40-94 dB (6 points) & ±3 dB at all points & |ESP8266 - Ref| < 3 dB \\
VAL-2 & Latency: hand clap → dashboard & <120 seconds & 95\% events < 120s \\
VAL-3 & Threshold: 40 dB, 60 dB (10 min each) & Correct classification & 0 false pos, 0 false neg \\
VAL-4 & Transient: 5× door slams & No alerts & 0 violation flags \\
VAL-5 & Sustained: 70 dB, 8 min & Alert after 5 min & Alert fires 5:00-5:30 \\
VAL-6 & Multi-sensor: same 60 dB tone & Agreement ±5 dB & All within 5 dB range \\
VAL-7 & Uptime: 48 hours continuous & >90\% & Gaps < 4.8 hr total \\
VAL-8 & Packet delivery: 24 hours & >99\% & Received/sent > 0.99 \\
VAL-9 & Coverage: 1m, 3m, 5m tests & Matches model ±3 dB & Measured vs predicted \\
VAL-10 & Baseline: empty room 3-6 AM & <45 dB & 95th percentile < 45 dB \\
\bottomrule
\end{tabular}
\end{table}

\subsection{Project Timeline}

\begin{table}[htbp]
\centering
\caption{10-Week Implementation Schedule}
\label{tab:timeline}
\begin{tabular}{@{}cp{5.5cm}p{2cm}p{2cm}@{}}
\toprule
\textbf{Week} & \textbf{Task} & \textbf{Duration} & \textbf{Risk} \\
\midrule
1 & Order components (shipping) & 7 days & Medium \\
2 & Assembly + power test & 3 days & Low \\
3-4 & Firmware development (ADC + WiFi) & 1 week & Medium \\
4 & Calibration (6-point procedure) & 1 day & Low \\
5-6 & Validation testing (VAL-1 to VAL-10) & 2 weeks & Low \\
7 & Continuous operation (48hr minimum) & 1 week & Low \\
8 & Data analysis + results documentation & 1 week & Low \\
9 & Technical report writing & 1 week & Low \\
10 & Final presentation prep + buffer & 1 week & Low \\
\midrule
\multicolumn{2}{l}{\textbf{Critical path: Weeks 1-4}} & \multicolumn{2}{c}{\textbf{Buffer: 2 weeks}} \\
\bottomrule
\end{tabular}
\end{table}

% ============================================================================
% SECTION 9: FEASIBILITY VERIFICATION
% ============================================================================
\section{Design Feasibility Verification}
\label{sec:feasibility}

Before committing to implementation, quantitative feasibility checks verify all requirements are achievable.

\subsection{Feasibility Checks Summary}

\begin{table}[htbp]
\centering
\caption{Feasibility Assessment Results}
\label{tab:feasibility}
\begin{tabular}{@{}p{4.5cm}p{2cm}p{5cm}@{}}
\toprule
\textbf{Check} & \textbf{Status} & \textbf{Quantitative Result} \\
\midrule
Detect 45 dB (SNR) & ✓ Pass & 9 dB margin above noise floor \\
Discriminate 45 vs 60 dB & ✓ Pass & 15 dB separation >> 6 dB error \\
Room coverage (5m) & ⚠ Pass & 1 dB margin worst-case (marginal) \\
Network bandwidth & ✓ Pass & 0.0077\% WiFi utilization \\
Power budget & ✓ Pass & 18\% USB capacity, 5.7× margin \\
Cost constraint & ✓ Pass & \$124 core < \$200 budget \\
Schedule constraint & ✓ Pass & 8-week critical path + 2-week buffer \\
\midrule
\textbf{Overall Decision} & \textbf{✓ GO} & \textbf{Proceed to implementation} \\
\bottomrule
\end{tabular}
\end{table}

\textbf{Risk identified:} Room coverage marginal at 5m distance (1 dB margin). Mitigation: Most violations occur <3m from sensor (central seating area) providing 5.5 dB margin. Monitor during testing; add second sensor if coverage inadequate.

% ============================================================================
% CONCLUSION
% ============================================================================
\section{Conclusion}

This PES defines an IoT-based acoustic monitoring system for institutional quiet zones. Key design decisions:

\begin{itemize}[noitemsep]
    \item \textbf{MAX9814 analog microphone:} Implementation simplicity prioritized over marginal SNR improvement (schedule risk management)
    \item \textbf{ESP8266 platform:} Cost-effective (\$6) with adequate processing capability for RMS calculation
    \item \textbf{Star topology with WiFi:} Leverages existing infrastructure, eliminates \$150 gateway cost
    \item \textbf{1 kHz sampling:} Sufficient for amplitude measurement despite violating Nyquist for frequency analysis
    \item \textbf{±3 dB accuracy target:} Minimum acceptable for discrimination (45 vs 60 dB), maximum achievable within \$200 budget
\end{itemize}

All feasibility checks pass. System ready for implementation with identified risks (glass room baseline, 5m coverage) and mitigation plans documented.

Total estimated cost: \$124 (core system), \$164 (with optional components). Budget remaining: \$36-76 for contingencies.

Timeline: 10 weeks with 2-week buffer. Critical path: component delivery + firmware development (Weeks 1-4).

\end{document}